
\begin{otherlanguage}{english}
\subsection*{Abstract}
\labSec{abstract_eng}
In recent times, there has been an increase in the relationship between Machine Learning and Data Communication. Clustering algorithms have some practical applications in Data Communication. There are times when data is to be transmitted from a sender to a receiver and the data is to be compressed where the compressed data should contain as much details as contained in the original information. Clustering is quite similar to compression, so it can be applied in Data Communication. One of such application is the Information Bottleneck Method. The Information Bottleneck seeks to pass data through a compact bottleneck representation. There are several algorithms used to implement the method but this report focuses on two of the algorithms which are Iterative and Agglomerative IB algorithms. The Iterative IB employs an iterative method that iterates over fixed point equations until a convergence criterion is reached. On the other hand, the Agglomerative IB employs a greedy clustering technique to find a clustering tree that goes in a bottom-up fashion.
\end{otherlanguage}

