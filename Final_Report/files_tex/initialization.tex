% IEEE packages ---------------------------------------------------------------------

% Some very useful LaTeX packages include:
% (uncomment the ones you want to load)


% *** MISC UTILITY PACKAGES ***
%
%\usepackage{ifpdf}
% Heiko Oberdiek's ifpdf.sty is very useful if you need conditional
% compilation based on whether the output is pdf or dvi.
% usage:
% \ifpdf
%   % pdf code
% \else
%   % dvi code
% \fi
% The latest version of ifpdf.sty can be obtained from:
% http://www.ctan.org/tex-archive/macros/latex/contrib/oberdiek/
% Also, note that IEEEtran.cls V1.7 and later provides a builtin
% \ifCLASSINFOpdf conditional that works the same way.
% When switching from latex to pdflatex and vice-versa, the compiler may
% have to be run twice to clear warning/error messages.






% *** CITATION PACKAGES ***
%
\usepackage{cite}
% cite.sty was written by Donald Arseneau
% V1.6 and later of IEEEtran pre-defines the format of the cite.sty package
% \cite{} output to follow that of IEEE. Loading the cite package will
% result in citation numbers being automatically sorted and properly
% "compressed/ranged". e.g., [1], [9], [2], [7], [5], [6] without using
% cite.sty will become [1], [2], [5]--[7], [9] using cite.sty. cite.sty's
% \cite will automatically add leading space, if needed. Use cite.sty's
% noadjust option (cite.sty V3.8 and later) if you want to turn this off.
% cite.sty is already installed on most LaTeX systems. Be sure and use
% version 4.0 (2003-05-27) and later if using hyperref.sty. cite.sty does
% not currently provide for hyperlinked citations.
% The latest version can be obtained at:
% http://www.ctan.org/tex-archive/macros/latex/contrib/cite/
% The documentation is contained in the cite.sty file itself.






% *** GRAPHICS RELATED PACKAGES ***
%
\usepackage{graphicx}
%add path for PC
\graphicspath{{/Users/obinnaisiwekpeni/Desktop/Information-Bottleneck/Final_Report/files_graphics/}}
%{J:/Masterarbeit/Latex/MA_template/files_graphics/}
% *** MATH PACKAGES ***
%
\usepackage[cmex10]{amsmath}
\usepackage{amssymb}
\usepackage{mathtools}
% A popular package from the American Mathematical Society that provides
% many useful and powerful commands for dealing with mathematics. If using
% it, be sure to load this package with the cmex10 option to ensure that
% only type 1 fonts will utilized at all point sizes. Without this option,
% it is possible that some math symbols, particularly those within
% footnotes, will be rendered in bitmap form which will result in a
% document that cannot be IEEE Xplore compliant!
%
% Also, note that the amsmath package sets \interdisplaylinepenalty to 10000
% thus preventing page breaks from occurring within multiline equations. Use:
%\interdisplaylinepenalty=2500
% after loading amsmath to restore such page breaks as IEEEtran.cls normally
% does. amsmath.sty is already installed on most LaTeX systems. The latest
% version and documentation can be obtained at:
% http://www.ctan.org/tex-archive/macros/latex/required/amslatex/math/





% *** SPECIALIZED LIST PACKAGES ***
%


%%%%%%%%%%%%%%%%%%%%%%%%%%%%%%%%%%%%%%%%%%%%%%%%%%%%%%%%%%%%%%
\usepackage[german,linesnumbered,boxed,hangingcomment]{algorithm2e}
\SetKwRepeat{Tue}{Tue}{solange}% define do while loop
\SetKwComment{Comment}{$\triangleright$\ }{} %change comment style
%example:
%\Tue{Endbedingung}{
     % mache diese Sachen\;
 %   }
%%%%%%%%%%%%%%%%%%%%%%%%%%%%%%%%%%%%%%%%%%%%%%%%%%%%%%%%%%%%%%

% *** ALIGNMENT PACKAGES ***
%
%\usepackage{array}
% Frank Mittelbach's and David Carlisle's array.sty patches and improves
% the standard LaTeX2e array and tabular environments to provide better
% appearance and additional user controls. As the default LaTeX2e table
% generation code is lacking to the point of almost being broken with
% respect to the quality of the end results, all users are strongly
% advised to use an enhanced (at the very least that provided by array.sty)
% set of table tools. array.sty is already installed on most systems. The
% latest version and documentation can be obtained at:
% http://www.ctan.org/tex-archive/macros/latex/required/tools/


%\usepackage{mdwmath}
%\usepackage{mdwtab}
% Also highly recommended is Mark Wooding's extremely powerful MDW tools,
% especially mdwmath.sty and mdwtab.sty which are used to format equations
% and tables, respectively. The MDWtools set is already installed on most
% LaTeX systems. The lastest version and documentation is available at:
% http://www.ctan.org/tex-archive/macros/latex/contrib/mdwtools/


% IEEEtran contains the IEEEeqnarray family of commands that can be used to
% generate multiline equations as well as matrices, tables, etc., of high
% quality.


%\usepackage{eqparbox}
% Also of notable interest is Scott Pakin's eqparbox package for creating
% (automatically sized) equal width boxes - aka "natural width parboxes".
% Available at:
% http://www.ctan.org/tex-archive/macros/latex/contrib/eqparbox/





% *** SUBFIGURE PACKAGES ***
%\usepackage[tight,footnotesize]{subfigure}
\usepackage{subfigure} 
% subfigure.sty was written by Steven Douglas Cochran. This package makes it
% easy to put subfigures in your figures. e.g., "Figure 1a and 1b". For IEEE
% work, it is a good idea to load it with the tight package option to reduce
% the amount of white space around the subfigures. subfigure.sty is already
% installed on most LaTeX systems. The latest version and documentation can
% be obtained at:
% http://www.ctan.org/tex-archive/obsolete/macros/latex/contrib/subfigure/
% subfigure.sty has been superceeded by subfig.sty.



%\usepackage[caption=false]{caption}
%\usepackage[font=footnotesize]{subfig}
% subfig.sty, also written by Steven Douglas Cochran, is the modern
% replacement for subfigure.sty. However, subfig.sty requires and
% automatically loads Axel Sommerfeldt's caption.sty which will override
% IEEEtran.cls handling of captions and this will result in nonIEEE style
% figure/table captions. To prevent this problem, be sure and preload
% caption.sty with its "caption=false" package option. This is will preserve
% IEEEtran.cls handing of captions. Version 1.3 (2005/06/28) and later
% (recommended due to many improvements over 1.2) of subfig.sty supports
% the caption=false option directly:
%\usepackage[caption=false,font=footnotesize]{subfig}
%
% The latest version and documentation can be obtained at:
% http://www.ctan.org/tex-archive/macros/latex/contrib/subfig/
% The latest version and documentation of caption.sty can be obtained at:
% http://www.ctan.org/tex-archive/macros/latex/contrib/caption/




% *** FLOAT PACKAGES ***
%
%\usepackage{fixltx2e}
% fixltx2e, the successor to the earlier fix2col.sty, was written by
% Frank Mittelbach and David Carlisle. This package corrects a few problems
% in the LaTeX2e kernel, the most notable of which is that in current
% LaTeX2e releases, the ordering of single and double column floats is not
% guaranteed to be preserved. Thus, an unpatched LaTeX2e can allow a
% single column figure to be placed prior to an earlier double column
% figure. The latest version and documentation can be found at:
% http://www.ctan.org/tex-archive/macros/latex/base/



%\usepackage{stfloats}
% stfloats.sty was written by Sigitas Tolusis. This package gives LaTeX2e
% the ability to do double column floats at the bottom of the page as well
% as the top. (e.g., "\begin{figure*}[!b]" is not normally possible in
% LaTeX2e). It also provides a command:
%\fnbelowfloat
% to enable the placement of footnotes below bottom floats (the standard
% LaTeX2e kernel puts them above bottom floats). This is an invasive package
% which rewrites many portions of the LaTeX2e float routines. It may not work
% with other packages that modify the LaTeX2e float routines. The latest
% version and documentation can be obtained at:
% http://www.ctan.org/tex-archive/macros/latex/contrib/sttools/
% Documentation is contained in the stfloats.sty comments as well as in the
% presfull.pdf file. Do not use the stfloats baselinefloat ability as IEEE
% does not allow \baselineskip to stretch. Authors submitting work to the
% IEEE should note that IEEE rarely uses double column equations and
% that authors should try to avoid such use. Do not be tempted to use the
% cuted.sty or midfloat.sty packages (also by Sigitas Tolusis) as IEEE does
% not format its papers in such ways.


%\ifCLASSOPTIONcaptionsoff
%  \usepackage[nomarkers]{endfloat}
% \let\MYoriglatexcaption\caption
% \renewcommand{\caption}[2][\relax]{\MYoriglatexcaption[#2]{#2}}
%\fi
% endfloat.sty was written by James Darrell McCauley and Jeff Goldberg.
% This package may be useful when used in conjunction with IEEEtran.cls'
% captionsoff option. Some IEEE journals/societies require that submissions
% have lists of figures/tables at the end of the paper and that
% figures/tables without any captions are placed on a page by themselves at
% the end of the document. If needed, the draftcls IEEEtran class option or
% \CLASSINPUTbaselinestretch interface can be used to increase the line
% spacing as well. Be sure and use the nomarkers option of endfloat to
% prevent endfloat from "marking" where the figures would have been placed
% in the text. The two hack lines of code above are a slight modification of
% that suggested by in the endfloat docs (section 8.3.1) to ensure that
% the full captions always appear in the list of figures/tables - even if
% the user used the short optional argument of \caption[]{}.
% IEEE papers do not typically make use of \caption[]'s optional argument,
% so this should not be an issue. A similar trick can be used to disable
% captions of packages such as subfig.sty that lack options to turn off
% the subcaptions:
% For subfig.sty:
% \let\MYorigsubfloat\subfloat
% \renewcommand{\subfloat}[2][\relax]{\MYorigsubfloat[]{#2}}
% For subfigure.sty:
% \let\MYorigsubfigure\subfigure
% \renewcommand{\subfigure}[2][\relax]{\MYorigsubfigure[]{#2}}
% However, the above trick will not work if both optional arguments of
% the \subfloat/subfig command are used. Furthermore, there needs to be a
% description of each subfigure *somewhere* and endfloat does not add
% subfigure captions to its list of figures. Thus, the best approach is to
% avoid the use of subfigure captions (many IEEE journals avoid them anyway)
% and instead reference/explain all the subfigures within the main caption.
% The latest version of endfloat.sty and its documentation can obtained at:
% http://www.ctan.org/tex-archive/macros/latex/contrib/endfloat/
%
% The IEEEtran \ifCLASSOPTIONcaptionsoff conditional can also be used
% later in the document, say, to conditionally put the References on a
% page by themselves.


% *** PDF, URL AND HYPERLINK PACKAGES ***
%
%\usepackage{url}
% url.sty was written by Donald Arseneau. It provides better support for
% handling and breaking URLs. url.sty is already installed on most LaTeX
% systems. The latest version can be obtained at:
% http://www.ctan.org/tex-archive/macros/latex/contrib/misc/
% Read the url.sty source comments for usage information. Basically,
% \url{my_url_here}.


% *** Do not adjust lengths that control margins, column widths, etc. ***
% *** Do not use packages that alter fonts (such as pslatex).         ***
% There should be no has to do such things with IEEEtran.cls V1.6 and later.
% (Unless specifically asked to do so by the journal or conference you plan
% to submit to, of course. )

%base packages

%+++++++++++++++++++++++++++++++++++++++++++++++++++++++++++++++++++++++++++++++++++++++++++++++++++++++++++++++++++++++
% correct bad hyphenation here
%\addto{\captionsenglish}{
	%\renewcommand{\refname}{Bibliography} %for article class
	%\renewcommand{\bibname}{Bibliography} %for book class
%}
%\hyphenation{min-i-mi-za-tion
%}

%+++++++++++++++++++++++++++++++++++++++++++++++++++++++++++++++++++++++++++++++++++++++++++++++++++++++++++++++++++++++
% additional math-packages
\usepackage{trfsigns}
\usepackage[thinspace,thinqspace,squaren,textstyle]{SIunits}
\usepackage{bm}
\usepackage{amssymb}
\usepackage{amsmath}

%\usepackage{mathtools}
%\mathtoolsset{showonlyrefs} % use \refeq{eqlabel} or \eqref{eqlabel} to reference equations. Then only really referenced equations get a number.

%+++++++++++++++++++++++++++++++++++++++++++++++++++++++++++++++++++++++++++++++++++++++++++++++++++++++++++++++++++++++
% include geometry-packages 
\usepackage[
  a4paper, % = 210mm x 297mm
  twoside, %PAGESTYLE
  left=22mm,
  right=22mm,
  top=33mm,
  bottom=30mm,
  bindingoffset=10mm,
  headheight=15mm
 % hoffset=5mm,          % (1)  (see latex_help\layout.png )  
 % voffset=10mm,         % (2)
 % headheight=10mm,      % (5)
 % headsep=10mm,         % (6)
 % textheight=240mm,     % (7)
 % textwidth=160mm,      % (8)
 % marginparsep=10mm,    % (9)
 % marginparwidth=12mm,  % (10)
 % footskip=10mm,        % (11)
]{geometry}

%\setlength{\baselineskip}{17pt}
\renewcommand{\baselinestretch}{1} % scales the value of \baselineskip. Its default value is 1.0
%\setlength{\parskip}{\baselineskip} % distance between paragraphs
%\setlength{\parindent}{0pt} % removes indent of paragraphs

%+++++++++++++++++++++++++++++++++++++++++++++++++++++++++++++++++++++++++++++++++++++++++++++++++++++++++++++++++++++++
 %package to define headings 
\usepackage[automark]{scrpage2}
%define headings:
\pagestyle{scrheadings}
\clearscrheadfoot

%\ohead{\pagemark}
\automark[subsection]{section}
\ofoot{\pagemark}
%\ihead{Name}

%+++++++++++++++++++++++++++++++++++++++++++++++++++++++++++++++++++++++++++++++++++++++++++++++++++++++++++++++++++++++
% table packages
\usepackage{multirow}
\usepackage{longtable}	%enable support for tables over 2 pages
\usepackage[table]{xcolor} %highlight cells in a table
\usepackage{tabularx}
\newcommand{\cellcolorA}{\cellcolor[gray]{0.9}} % Highlight Cell - command
\newcommand{\cellcolorB}{\cellcolor[gray]{0.8}} % Highlight Cell - command
\newcommand{\cellcolorC}{\cellcolor[gray]{0.7}} % Highlight Cell - command
\newcommand{\cellcolorD}{\cellcolor[gray]{0.6}} % Highlight Cell - command

%+++++++++++++++++++++++++++++++++++++++++++++++++++++++++++++++++++++++++++++++++++++++++++++++++++++++++++++++++++++++
% include psfrag (replace text in figures)
\usepackage{psfrag}
\usepackage{placeins} %placeins ist ein Paket, welches verhindert, dass Floats hinter dem Befehl \FloatBarrier erscheint. 

%+++++++++++++++++++++++++++++++++++++++++++++++++++++++++++++++++++++++++++++++++++++++++++++++++++++++++++++++++++++++
% include package for new algorithm-environment
%\usepackage[Alg.,nothing]{algorithm} %25.09.2014
\usepackage{caption}%[2007/09/01]

%+++++++++++++++++++++++++++++++++++++++++++++++++++++++++++++++++++++++++++++++++++++++++++++++++++++++++++++++++++++++
%change labels
%\renewcommand{\listfigurename}{List of Figures}
%\renewcommand{\listtablename}{List of Tables}
\renewcommand{\figurename}{Fig.}
\renewcommand{\tablename}{Tab.}
\addto\captionsenglish{%
	\renewcommand{\figurename}{Fig.}
	\renewcommand{\tablename}{Tab.}
}
\newcommand{\sectionname}{Sec.}
%\newcommand{\equationname}{Eq.}

\newcommand{\etal}[0]{\textit{et al.}}
\newcommand{\refFig}[1]{Fig.~\ref{#1}}
\newcommand{\refAlg}[1]{Alg.~\ref{#1}}
\newcommand{\refPag}[1]{p.~\pageref{#1}}
\newcommand{\refApp}[1]{App., part~\ref{app:#1}}
\newcommand{\refTab}[1]{Tab.~\ref{#1}}
\newcommand{\refSec}[1]{Sec.~\ref{#1}}
\newcommand{\refCha}[1]{Ch.~\ref{cha:#1}}
\newcommand{\refEq}[1]{\eqref{#1}}

\newcommand{\labFig}[1]{\label{fig:#1}}
\newcommand{\labAlg}[1]{\label{alg:#1}}
\newcommand{\labApp}[1]{\label{app:#1}}
\newcommand{\labTab}[1]{\label{tab:#1}}
\newcommand{\labSec}[1]{\label{sec:#1}}
\newcommand{\labCha}[1]{\label{cha:#1}}
\newcommand{\labEq}[1]{\label{eq:#1}}

\numberwithin{equation}{chapter} %add section number to equations (e.g. "(1.4)" instead of "(4)")
\numberwithin{figure}{chapter}
\numberwithin{table}{chapter}
%\numberwithin{algorithm}{chapter}%25.09

%Redefine \dotfill (larger space between dots)
\makeatletter
\renewcommand*\dotfill{\leavevmode%
	\leaders\hbox{$\m@th
		\mkern \@dotsep mu\hbox{.}\mkern \@dotsep
		mu$}\hfill\kern\z@}
\makeatother
		
%+++++++++++++++++++++++++++++++++++++++++++++++++++++++++++++++++++++++++++++++++++++++++++++++++++++++++++++++++++++++
% Widows and orphans in LaTeX
% When writing in LaTeX, you sometimes experience that words or short lines of text at the beginning or end of a paragraph are printed at the bottom or top of a page. 
% This makes them separated from the rest of the text in the paragraph. Such words or lines are called widows and orphans.
% Orphans are the first words/lines of a paragraph at the end of a page and widows are the last words/lines that appear at the page top.
% Here, is a way to change the way LaTeX deal with these phenomena.
%
% The LaTeX algorithm takes these artifacts into account, but not always in a preferable way. 
% It makes use of \widowpenalty and \clubpenalty. By default, these parameters are set to a value of 150.
% Increasing these numbers, to for instance 500, makes them more important for LaTeX:
\clubpenalty=150000 
\widowpenalty=150000

%+++++++++++++++++++++++++++++++++++++++++++++++++++++++++++++++++++++++++++++++++++++++++++++++++++++++++++++++++++++++
% bibliography style


%\bibliographystyle{IEEEtran}
\bibliographystyle{alpha}

% remove default caption
%\renewcommand{\bibname}{}
%\renewcommand{\refname}{}

%+++++++++++++++++++++++++++++++++++++++++++++++++++++++++++++++++++++++++++++++++++++++++++++++++++++++++++++++++++++++
%hyperlink support 
\usepackage[
      pdfauthor={Autorenname}, 
      pdftitle={Titel der Arbeit}, 
      colorlinks=true, 
      urlcolor=black, 
      linkcolor=black,
      citecolor=black,
      pdfpagelayout=TwoPageRight,	%TwoPageRight/SinglePage %PAGESTYLE
      bookmarksnumbered,			% bookmarks in the pdf-viewer will be numbered
      bookmarksopen,				% bookmarks will be open when pdf-viewer is started
      plainpages=false,				% prevents "destination with the same identifier"-Warning
      linktocpage=true,
      %breaklinks=true
]{hyperref}

%\usepackage{breakurl}  % When you do not compile with pdflatex (e.g. because you want to use \usepackage{psfrag}), urls with hyperref are not line wrapped. This package wraps them again and remains working lins! Great particularly for bibliographies with links.

%+++++++++++++++++++++++++++++++++++++++++++++++++++++++++++++++++++++++++++++++++++++++++++++++++++++++++++++++++++++++
\usepackage{acronym}
% TexWorks hat genau hier auf einmal rumgemeckert
%\newcommand{\bflabel}[1]{\normalfont{\normalsize{#1}}\hfill} %set font of abbreviations to normal font

%% glossaries (abbreviations and symbols)
%\usepackage[
%	nonumberlist,		%do not show page numbers
%	acronym,			%list of abbreviations
%	toc,				%add labes in table of content
%	section,			%move labes in toc to same level as sections
%	hyperfirst=false	%suppress the first use link for all terms
%]{glossaries}
%
%\newglossary[slg]{symbols}{syi}{syg}{Symbols}
%\renewcommand*{\glspostdescription}{} %remove dot at the end of each description
%\makeglossaries % create glossary-files

%+++++++++++++++++++++++++++++++++++++++++++++++++++++++++++++++++++++++++++++++++++++++++++++++++++++++++++++++++++++++++
%andere Pakete
%\usepackage{amsfonts}  
%Tikz matlab figures
\usepackage{pdfpages} 
\usepackage{tikz,pgfplots}
\usetikzlibrary{calc}

%\pgfplotsset{compat=newest}
%\usepgfplotslibrary{external} 
%\tikzexternalize[prefix=TikzPictures/]
%

\usepackage{color}
\definecolor{uniblue}{rgb}{0,0.29,0.6} % könnte Probleme machen, wenn noch xcolor eingebunden ist!!!!

\usepackage{forest}
%\usetikzlibrary{positioning}
%\usetikzlibrary{trees,arrows,positioning}
\usetikzlibrary{arrows,chains,matrix,positioning,scopes}

%\tikzset{level 1/.style={level distance=3cm, sibling distance=2.5cm}}
%\tikzset{level 2/.style={level distance=4cm, sibling distance=2cm}}

%\tikzset{bag/.style={circle}}

\usepackage{rotating} 
\usepackage{longtable} 

\makeatletter
\tikzset{join/.code=\tikzset{after node path={%
\ifx\tikzchainprevious\pgfutil@empty\else(\tikzchainprevious)%
edge[every join]#1(\tikzchaincurrent)\fi}}}
\makeatother
%
\tikzset{>=stealth',every on chain/.append style={join},
         every join/.style={->}}
\tikzstyle{labeled}=[execute at begin node=$\scriptstyle,
   execute at end node=$]
%
%noch mehr andere Pakete

%Für große griechische Buchstaben wie ein großes Lambda 8.02.2016
\usepackage{ upgreek }
%für doppelt angestrichene Buchstaben für reelle, komplexe Zahlen, Erwartungswert
\usepackage{dsfont}